% !TEX encoding = UTF-8 Unicode
% !TEX TS-program = xelatex

\documentclass[a4paper,11pt]{article}

% Στο παρακάτω αρχείο έχουμε διάφορους ορισμούς. Δεν χρειάζεται να αλλάξετε κάτι σε αυτό.
% !TEX encoding = UTF-8 Unicode
% !TEX root =  main.tex
\usepackage{letter}
\usepackage{colortbl}
\usepackage{soul}
\usepackage{xltxtra} 
\usepackage{xgreek} 
\usepackage{fontspec}
\usepackage{listings}
\usepackage{graphicx}
\usepackage{fancyhdr}

% Header and Footer
\setlength{\headheight}{15.2pt}
\pagestyle{fancyplain}
\fancyhf{}
\lhead{ \fancyplain{\footnotesize{1ο Φοιτητικό Συνέδριο Πληροφορικής \& Τηλεπικοινωνιών}}{\footnotesize{1ο Φοιτητικό Συνέδριο Πληροφορικής \& Τηλεπικοινωνιών}}}
\rhead{ \fancyplain{\footnotesize{Πανεπιστήμιο Πελοποννήσου, Τρίπολη}}{\footnotesize{Πανεπιστήμιο Πελοποννήσου, Τρίπολη}} }
%\lfoot{ \fancyplain{}{\footnotesize{11 Ιουνίου 2025}} }
\rfoot{ \fancyplain{\thepage}{\thepage} }

%
% Tables
\usepackage{colortbl}
\usepackage[table]{xcolor}
\definecolor{lightgray}{gray}{0.9}

%
% Page size
\usepackage[margin=2.5cm]{geometry}
% \setlength{\textheight}{23cm}
% \setlength{\textwidth}{15.5cm}
% \setlength{\oddsidemargin}{0.2cm}
% \setlength{\evensidemargin}{0.2cm}
% \setlength{\topmargin}{-1.2cm}
% %\setlength{\headsep}{1.5cm}
% \setlength{\footskip}{2cm} % space between page number and text

% 1.2 spacing
 %\renewcommand{\baselinestretch}{1.1}

%
% Fonts sets
\setmainfont[ItalicFont=cmunsl.ttf,BoldFont=cmunbx.ttf,SmallCapsFont=cmunbx.ttf]{cmunrm.ttf}  
%\setmainfont[Mapping=tex-text]{CMU Serif} 
%\setmainfont[Mapping=tex-text]{Marion} 
%\setmainfont[Mapping=tex-text]{Athelas} 

\setmonofont{Courier New}
%\setmonofont{Tahoma}

\frenchspacing

%
% Definitions
\newtheorem{definition}{Ορισμός}
\newtheorem{theorem}{Θεώρημα}
\newenvironment{proof}{\noindent{\bf Απόδειξη:}}{\rule[-0.25mm]{0.25cm}{0.25cm}\vspace{1.5ex}}
\newcommand{\proofEnd}{\ \rule[-0.25mm]{0.25cm}{0.25cm}}
\newcommand{\defEnd}{\ \rule[-0.25mm]{0.25cm}{0.25cm}}

%
% Algorithms
\usepackage[figure,vlined,linesnumbered]{algorithm2e}
\SetKwInOut{Input}{Input}
\SetKwInOut{Output}{Output}
\SetKwInOut{Parameter}{Parameter}
\SetKwInput{Algorithm}{Algorithm}
\SetKwInput{Remark}{Remark}
\SetKw{Let}{Let}
\SetKw{Add}{Add}
\SetKwComment{Comment}{{// }}{}

%
% References
% \usepackage[colorlinks,linkcolor=blue,linktocpage]{hyperref}
\usepackage[colorlinks,linkcolor=blue,citecolor=blue,urlcolor=blue,linktocpage]{hyperref}

% Code
\definecolor{customgreen}{rgb}{0,0.6,0}
\definecolor{customgray}{rgb}{0.5,0.5,0.5}
\definecolor{custommauve}{rgb}{0.6,0,0.8}
\lstset{ 
  basicstyle=\small\ttfamily,%,        % the size of the fonts that are used for the code
  breaklines=true,                 % sets automatic line breaking
  commentstyle=\color{customgreen},    % comment style
  firstnumber=1,                % start line enumeration with line 1000
  frame=single,	                   % adds a frame around the code
  keepspaces=true,                 % keeps spaces in text, useful for keeping indentation of code (possibly needs columns=flexible)
  keywordstyle=\color{blue},       % keyword style
  numbers=left,                    % where to put the line-numbers; possible values are (none, left, right)
  numbersep=10pt,                   % how far the line-numbers are from the code
  numberstyle=\tiny\color{customgray}, % the style that is used for the line-numbers
  rulecolor=\color{black},         % if not set, the frame-color may be changed on line-breaks within not-black text (e.g. comments (green here))
  showspaces=false,                % show spaces everywhere adding particular underscores; it overrides 'showstringspaces'
  showstringspaces=false,          % underline spaces within strings only
  showtabs=false,                  % show tabs within strings adding particular underscores
  stepnumber=1,                    % the step between two line-numbers. If it's 1, each line will be numbered
  stringstyle=\color{custommauve},     % string literal style
  tabsize=2,	                   % sets default tabsize to 2 spaces
  aboveskip=-10pt,% 
  belowskip=10pt,%
  title=\lstname                   % show the filename of files included with \lstinputlisting; also try caption instead of title
}

%
% SQL
\lstnewenvironment{SQLcode}{%
\lstset{language=SQL,%
%keywordstyle=\bf\sffamily\color{blue},%
%commentstyle=\color{ForestGreen},%
%stringstyle=\color{Maroon},%
%basicstyle=\ttfamily\normalsize\color{black},%
%frame=lines,%
showspaces=false,% 
showstringspaces=false,%
tabsize=4,% 
%aboveskip=-10pt,% 
%belowskip=10pt,%
lineskip=2pt%
% numbers=left, numberstyle=\tiny, stepnumber=1, numbersep=5pt, numberblanklines=false, %
% breaklines, breakatwhitespace, prebreak=\_, breakindent=0pt %
}}
{}

%
% LaTeX
\lstnewenvironment{latex}{%
\lstset{language=TeX,%
keywordstyle=\bf\sffamily\color{blue},%
commentstyle=\color{ForestGreen},%
stringstyle=\color{Maroon},%
basicstyle=\ttfamily\normalsize\color{black},%
%frame=lines,%
showspaces=false,% 
showstringspaces=false,%
tabsize=4,% 
aboveskip=10pt,% 
belowskip=10pt,%
lineskip=2pt%
% numbers=left, numberstyle=\tiny, stepnumber=1, numbersep=5pt, numberblanklines=false, %
% breaklines, breakatwhitespace, prebreak=\_, breakindent=0pt %
}}
{}

\bibliographystyle{plain}

% Ο τίτλος της εργασίας σας
\title{
\huge ΤΙΤΛΟΣ ΕΡΓΑΣΙΑΣ
}

% Ο/οι συγγραφείς
\author{Όνομα συγγραφέα, Τμήμα/Μεταπτυχιακό, Πανεπιστήμιο, Πόλη, email \\
Όνομα συγγραφέα, Τμήμα/Μεταπτυχιακό, Πανεπιστήμιο, Πόλη, email}

% Για να μην τυπωθεί η τρέχουσα ημερομηνία, αλλά εκείνη που επιθυμείτε 
\date{11 Ιουνίου 2025}

% Από εδώ και κάτω αρχίζει το κείμενό σας.
\begin{document}

\maketitle

% Με μια μόνο εντολή φτιάχνετε το πίνακα περιεχομένων, Μαγεία!
%\tableofcontents

%
%%
%
\begin{center}
\section*{Περίληψη/Abstract (έως 300 λέξεις)}
Κείμενο    
\[\]
\textbf{Λέξεις κλειδιά} (έως 6):
\end{center}

%
%%
%
\section{Εισαγωγή/Introduction}\label{sec:intro}
Το σύστημα επεξεργασίας κειμένων \emph{\LaTeX} \cite{Lamport86} είναι βασισμένο στο σύστημα \emph{\TeX} του \href{https://en.wikipedia.org/wiki/Donald_Knuth}{Donald Knuth}. Το μεγάλο πλεονέκτημα του {\LaTeX} είναι ότι ο χρήστης του επικεντρώνεται στο περιεχόμενο και όχι στην παρουσίασή του. 

Το πώς εμφανίζεται το περιεχόμενο καθορίζεται από το στυλ (class στην ορολογία του \LaTeX) που χρησιμοποιούμε και γενικά έχουμε ελάχιστη (συνήθως καμία) δυνατότητα μεταβολής του. Για το συνέδριο θα χρησιμοποιήσετε το στυλ κειμένου που σας έχει δοθεί.

%%
\subsection{Σημαντικές παρατηρήσεις}
Είναι χρήσιμο να γνωρίζετε τα εξής:
\begin{enumerate}
    \item Η αρίθμηση των ενοτήτων, υποενοτήτων, σχημάτων, ορισμών, θεωρημάτων, εικόνων κλπ.\ γίνεται αυτόματα και χωρίς λάθη. Κάνετε απλά χρήση των κατάλληλων εντολών όπως στο υπόδειγμα. Γράφετε για παράδειγμα το παρακάτω όταν ξεκινάτε νέα ενότητα: 
    \begin{verbatim}
       \section{Παραδείγματα χρήσης}
    \end{verbatim}
    \item Μπορείτε να αναφέρεστε σε ενότητες, υποενότητες, σχήματα, ορισμούς, θεωρήματα, εικόνες, κλπ. Οποιαδήποτε εισαγωγή ή αναφορά ενημερώνεται αυτόματα και διατηρείται συνεπές το κείμενο. Για επεξηγήσεις δείτε την Ενότητα~\ref{sec:descr}.
    \item Μπορείτε να γράψετε ορισμούς και θεωρήματα με πολύ εύκολο τρόπο, όπως στην Ενότητα~\ref{sec:theorem}.
    \item Μπορείτε να γράψετε πολύπλοκες μαθηματικές εκφράσεις με πολύ εύκολο τρόπο, να τις αντιγράψετε και να τις επαναχρησιμοποιήσετε. Δείτε για παράδειγμα την Ενότητα~\ref{sec:math}.  
    \item Μπορείτε να εισάγετε αλγορίθμους ή κώδικα και αυτόματα να χρωματίσει τις δεσμευμένες λέξεις, όπως φαίνεται στην Ενότητα~\ref{sec:alg}. 
\end{enumerate}

%
%
\subsection{Θεωρήματα}\label{sec:theorem}
Ορισμούς, θεωρήματα και αποδείξεις, γράφετε κάνοντας χρήση του κατάλληλου περιβάλλοντος.

\begin{theorem}
Με το \emph{\LaTeX} \cite{Lamport86} τα κείμενά σας είναι: 
\begin{enumerate}
\item
πιο όμορφα και 
\item
γράφονται πιο γρήγορα.
\end{enumerate}
\end{theorem}

\begin{proof}
Για να το (1) αρκεί να δείτε μια εργασία σε {\LaTeX} και την εργασία με το ίδιο περιεχόμενο σε ένα άλλο σύστημα επεξεργασίας κειμένου.  Το (2) θα το διαπιστώσετε αφού διαβάσετε αυτό το κείμενο και κάνετε την πρώτη σας εργασία. 
\end{proof}

%
%
\subsection{Μαθηματικά}\label{sec:math}
Μαθηματικά μπορείτε να γράψετε σε ξεχωριστή γραμμή:
\[ x^n + y^n = z^n \]
αλλά και μέσα στην παράγραφο $x^n + y^n = z^n$.

Αν θέλετε να χρησιμοποιήσετε αρίθμηση στις μαθηματικές εκφράσεις και να κάνετε αναφορά στο κείμενο, τότε κάντε χρήση κατάλληλου περιβάλλοντος όπως στην Εξίσωση~\ref{eq:sum}.

\begin{equation}
x^n + y^n = z^n
\label{eq:sum}    
\end{equation}

Μερικά ακόμα παραδείγματα χρήσης μαθηματικών:
\[\left\{
\begin{array}{r@{\:=\:}ll}
A & 1\\
B & 2\\
  f(x) & x^2\\
  g(x) & \frac{1}{x}\\
  F(x) & \int^a_b \frac{1}{3}x^3
\end{array}
\right.
\]

%
%
\subsection{Αλγόριθμοι} \label{sec:alg}
Ένα παράδειγμα είναι ο Αλγόριθμος \textsc{First} που παρουσιάζεται στο Σχήμα~\ref{alg:first}.

 \begin{algorithm}[t]
 \Algorithm{\textsc{First}}
 \scriptsize
 \BlankLine

 $T' =T$ \Comment{Initialize output}
 \For{$i=1$ \KwTo $m$} {
 a
 }

 \For(\tcp*[h]{test if $C'=C_i$}){$i=1$ to $t$}{ \label{alg:tCM:for}%
 a
 }
\caption{Ο πρώτος μου αλγόριθμος}
\label{alg:first}
 \end{algorithm}

 %
%
\subsubsection{Κώδικας}\label{sec:code}
Κώδικας σε C:
\begin{lstlisting}[language=C]
#include <stdio.h>
int main() {
   // printf() displays the string
   printf("Hello, World!");
   return 0;
}
\end{lstlisting}

%Κώδικας σε SQL:
%\begin{SQLcode}
%SELECT %max(diff([arrtime,arrday],[deptime,depday])),
	   %avg(diff([arrtime,arrday],[deptime,depday]))
%FROM route
%GROUP BY sid
%\end{SQLcode}

%Κώδικας σε python:
%\begin{lstlisting}[language=python]
%class MyClass(Yourclass):
%    def __init__(self, test):
%        self.test = test
%\end{lstlisting}

\subsection{Πίνακες}\label{sec:tables}
Οι πίνακες μπορούν να έχουν μορφή αντίστοιχη με εκείνη που έχει ο Πίνακας~\ref{tab:example}. Θυμηθείτε ότι οι πίνακες θα πρέπει να έχουν περιγραφεί μέσα στο κείμενο.

\begin{table}
\centering
\rowcolors{1}{}{lightgray}
\begin{tabular}{|c|r|l|}\hline
\rowcolor{gray}
Στο κέντρο & Δεξιά & Αριστερά \\\hline\hline
    1000 & 1000 & 1000 \\\hline
    100 & 100 & 100 \\ \hline
\end{tabular}
\caption{Προσέξτε την αυτόματη εναλλαγή χρωμάτων}
\label{tab:example}
\end{table}

%
%%
%
\section{Περιγραφή Προβλήματος/Problem Description}\label{sec:descr}

Στη δήλωση κάθε ενότητας, ορισμού, κ.λπ. μπορούμε να ορίσουμε μια πινακίδα, π.χ.:
\begin{verbatim}
\section{Παραδείγματα χρήσης}\label{sec:examples}
\end{verbatim}
και στο κείμενο να αναφερόμαστε σε αυτήν την ενότητα χρησιμοποιώντας την πινακίδα, π.χ.:
\begin{verbatim}
όπως είδαμε στην Ενότητα~\ref{sec:examples}.
\end{verbatim}

Σημαντικό είναι επίσης να ακολουθήσετε τις εξής οδηγίες:
\begin{enumerate}
    \item Όταν αναφέρεστε σε κεφάλαιο, ενότητα, σχήμα, πίνακα κ.λπ. και ακολουθείται από αριθμό το γράφετε με κεφαλαίο, π.χ., στο Κεφάλαιο 3, στην Ενότητα 3.2, στο Σχήμα 12. Αν δεν υπάρχει αριθμός το γράφετε με μικρό, π.χ., όπως είδαμε στο προηγούμενο κεφάλαιο.
    \item Κάνετε πάντα αναφορά στο κείμενο για τα σχήματα, τους πίνακες, κ.λπ. Δεν γράφετε δηλαδή «όπως βλέπουμε στην παρακάτω εικόνα» αλλά «όπως φαίνεται στο Σχήμα~\ref{fig:dit}».
\end{enumerate}

\begin{figure}
\centering
\includegraphics[scale=.2]{dit-uop-logo-regular-red}
\caption{Το λογότυπο του τμήματος}
\label{fig:dit}
\end{figure}


%
%
\subsection{Βιβλιογραφικές αναφορές}\label{sec:cite}
Όλες οι βιβλιογραφικές αναφορές γράφονται σε ένα  αρχείο, όπως το sample.bib. Κάθε αναφορά έχει έναν κωδικό, π.χ., Lamport86, και για να αναφερθείτε σε αυτή χρησιμοποιείτε την εντολή:
\begin{verbatim}
\cite{Lamport86}
\end{verbatim}
Για παράδειγμα, ο πατέρας του {\LaTeX} είναι ο L.\ Lamport \cite{Lamport86}.

%
%
\subsection{Σύνδεσμοι}
Μερικά παραδείγματα συνδέσμων: \url{http://www.latex-tutorial.com}, \href{mailto:sss@uop.gr}{email} και \href{https://teams.microsoft.com/}{teams}. Δείτε ότι τα χρώματα έχουν μπει αυτόματα! 

\bibliography{sample}

\end{document}
